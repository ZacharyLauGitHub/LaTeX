\documentclass[UTF8,11pt,titlepage,a4paper]{ctexart}
%必需项:文档类说明,一般ctexart即可,不一般的时候看你需要选择,[]里的是可选项,UTF8表示中英混排,11磅,标题单独成页,纸张大小a4纸。若不设置则为默认值


%以下是导言区,可引入宏包,进行一些个性化定制,以及写标题内容。
%以下为可选项,其中amsmath宏包最常用,其他宏包看需要,常见宏包TeXlive自带,不常见宏包需下载
\usepackage{amsmath}
%使用宏包,支持数学公式输入
\usepackage{hyperref}
%支持插入超连接
\usepackage{booktabs}
%支持插入三线表
\usepackage{graphics}
%支持插入图形
\usepackage{xcolor}
%支持改变字体色彩
\usepackage{amssymb}
%加载数学字体宏包
\usepackage{amsthm}
%加载宏包,可以插入证明



%下面是一些个性化定制
\renewcommand{\theenumi}{\roman{enumi}}
%改变有序列表编号为小写罗马数字
\renewcommand{\labelitemi}{\S}
%改变无序列表前的字符
\newtheorem{theorem}{定理}[section]
%重命名定理
\hypersetup{
	colorlinks=true,
	linkcolor=black,
	citecolor=black
}
%隐藏链接边框与颜色,设定引用为黑色


%以下是标题区
\title{这里是标题}
%标题
\date{\today}
%日期设为当日
\author{ 幻灭凌王\\ 
	$ \dagger $ Zhengzhou University $ \dagger $}
%作者及学校,\\是换行命令,注意区分\par分段命令与换行命令\\
%导言区结束


%以下是正文
\begin{document}                                      
	%document环境,必需项,所有想在正文中显示的必须放在document环境,标题区比较特殊,先写在导言区然后用\maketitle命令引入
   \maketitle
   %这里会显示标题信息
   	\begin{abstract}
   	%摘要
   	这里是摘要文字
   \end{abstract} 
	\setcounter{tocdepth}{4}
	%设定目录深度                      
	\tableofcontents
	%列出目录
	\newpage
	%手动分页
\part{这里是part}
\section{这里是section}
\subsubsection{这里是subsection}
\paragraph{这里是paragraph标题}这里是文字内容,这里是引用 \cite{AndersonMore}
%cite命令表示引用参考文献
\bibliographystyle{plain}
%引入参考文献格式
\bibliography{reference}
%引入的bib格式参考文献
\end{document}